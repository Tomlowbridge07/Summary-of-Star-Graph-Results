\documentclass[a4paper,10pt]{article}

\usepackage{color}
\usepackage{xcolor}
\usepackage{tikz}
\usepackage{amsmath}
\usepackage{amssymb}
\usepackage{amsthm}
\usepackage{graphicx}
\usepackage{mathtools}
\usepackage{wrapfig}
\usepackage{multirow}
\usepackage{comment}
\usepackage{natbib}
%\usepackage{float}
\usepackage{appendix}
\usepackage{subfig}
%\usepackage{enumitem}
\usepackage{newfloat}
\usepackage[utf8]{inputenc}
\usepackage{floatrow}
\usepackage{bm}

\usetikzlibrary{calc}
\usetikzlibrary{fit}
\usetikzlibrary{shapes.misc,calc, positioning, hobby, backgrounds}


%\DeclarePairedDelimiter{\floor}{\lfloor}{\rightfloor}
%\DeclarePairedDelimiter{\ceil}{\lceil}{\rceil}

\newcommand{\halflength}{\ensuremath{\floor{\frac{m}{2}}}}
\newcommand{\floor}[1]{\left \lfloor #1 \right \rfloor}
\newcommand{\ceil}[1]{\left \lceil #1 \right \rceil}

\newtheorem{theorem}{Theorem}[section]
\newtheorem{corollary}[theorem]{Corollary}
\newtheorem{lemma}[theorem]{Lemma}

\theoremstyle{definition}
\newtheorem{definition}[theorem]{Definition}

\theoremstyle{definition}
\newtheorem{example}[theorem]{Example}
 
\theoremstyle{remark}
\newtheorem*{remark}{Remark}

\theoremstyle{definition}
\newtheorem*{note}{Note}

\newcommand{\tightoverset}[2]{%
  \mathop{#2}\limits^{\vbox to -.1ex{\kern-0.75ex\hbox{$#1$}\vss}}}

\DeclareFloatingEnvironment[fileext=los,
    listname={List of Example Figures},
    name=Example Figure,
    placement=tbhp,
    within=section,]{examplefigure}

\title{Summary of Star Graph Results}
\date{\today}
\author{Thomas Lowbridge \\ School of Mathematical Sciences \\ University of Nottingham}

\bibliographystyle{plain}

\begin{document}

\pagestyle{empty}
{
  \renewcommand{\thispagestyle}[1]{}
  \maketitle
  \tableofcontents  
}
\clearpage
\pagestyle{plain}


\setlength{\parindent}{0pt}
\setlength{\parskip}{1em}

\newpage
\pagenumbering{arabic}
\section[]{Summary on $S_{n}^k$ results}

\subsection[]{Full Solution for $m \geq 2(k+1)$}
\begin{itemize}
\item The patroller follows a hamiltonian path giving the bound $V \geq \frac{m}{2(n+k)}$
\item The attacker follows the `time-delayed' attack (attacking at times $0,...,2k+1$ at elgonated node and times $k,k+1$ at normal external nodes) giving the bound $V \leq \frac{m}{2(n+k)}$
\end{itemize}

Hence we have $V= \frac{m}{2(n+k)}$.

State of proof: Needs reworking to be more accessible

\subsection[]{Full solution for $m=2k+1,2k$}
We further split this into two regions
\begin{itemize}
\item $m=2k$ or $m=2k+1$ and $m \geq 2n-1$:
 First it is worth noting that $R=\emptyset$ , $M=\emptyset$.
 \begin{itemize}
 \item The patroller follows the Combinatorial Improvement, giving the bound $V \leq \frac{\floor{\frac{m}{2}}}{\floor{\frac{m}{2}}+n-1} \frac{2k}{2k+2(n-1} = \frac{m}{m+2(n-1)}$
 \item The attacker `augments' their `time-delayed' attack on $m \geq 2(k+1)$ to place attacks at times $0,...,2k-1$ at the elongated node and times $k-1,k$ at normal external nodes, giving the bound $V \geq \frac{2k}{2k+2(n-1)}$
 \end{itemize}
 Hence we have $V=\frac{2k}{2k+2(n-1)}$.
 Noting: This means that for $m=2k \implies V=\frac{m}{m+2(n-1)}$.
 
 State of proof: Needs to be done

\item $m=2k+1$ and $m \leq 2(n-1)$:
 It is worth noting that from the above, the transition has happened as the attackers strategy is no longer true.
 \begin{itemize}
 \item The patroller will follow a decomposition into $S_{n-1}$ and $L_{k+1}$ giving $V \geq \frac{1}{\frac{1}{V(S_{n-1})}+\frac{1}{V(L_{k+1})}}=\frac{1}{1+\frac{2(n-1)}{2k+1}}=\frac{2k+1}{2k+1+2(n-1)}=\frac{m}{m+2(n-1)}$.
 \item The attacker will `augment' their `time-delayed' attack to placed attacks at times $0,...,2k$ at the elongated node and times $k,k+1$ (or $k-1,k$ is proposed to work just as well), giving the bound $V \leq \frac{2k+1}{2k+1+2(n-1)}=\frac{m}{m+2(n-1)}$.
 \end{itemize}
 Hence we have $V=\frac{m}{m+2(n-1)}$.
 Noting: This type of patroller strategy is just as good for the case of $m=2k$ when $m \leq 2(n-1)$ and gives the same bound.
 
 State of proof: Needs to be done

\end{itemize}

\subsection[]{Full solution for $m=2k-1,2k$}

We further split this into two regions

\begin{itemize}
\item $m=2k-2$ or $m=2k-1$ and $m \geq 2n-1$:

 \begin{itemize}
 \item The patroller can follow 


  \item The attacker will `augment' their `time-delayed' attack to place attacks at times $0,...,2k-3$ at the elongated node and times $k-2,k-1$ at normal external nodes. With the node along the line, which is extended by $k$ also being attacked at times $k-2,k-1$, giving the bound $V \leq \frac{m}{2n + 2k-2}=\frac{m}{2(n+k-1)}$
 \end{itemize}

\item $m=2k-1$ and $m \leq 2(n-1)$:

 \begin{itemize}
 \item The patroller will follow the decompostion into $S_{n-1}$ and $L_{k+1}$ giving $V \geq \frac{1}{\frac{1}{V(S_{n-1})}+\frac{1}{V(L_{k+1})}}=\frac{1}{\frac{2(n-1)}{m}+\frac{2k}{m}}=\frac{m}{2(n+k-1)}=\frac{2k-1}{2(n+k-1)}$
 \item The attacker will `augment' their `time-delayed' attack to place attacks at times $0,...,2k-3$ at the elongated node and times $k-2,k-1$ at normal external nodes. With the node along the line, which is extended by $k$ also being attacked at times $k-2,k-1$, giving the bound $V \leq \frac{m}{2n + 2k-2}=\frac{m}{2(n+k-1)}=\frac{2k-1}{2(n+k-1)}$
 \end{itemize}  
 Hence we have $V=\frac{m}{2(n+k-1)}$
 
\end{itemize}

\section[]{Summary on $S_{n}^{\bm{k}}$ results}

\subsection[]{Full Solution for $m \geq 2(k_{\max} +1)$}

\begin{itemize}
\item The patroller follows a hamiltonian path giving, $V \geq \frac{m}{2(n+\sum k_{i})}$
\item The attacker uses the `time-delayed' attack, attacking an external node which is $i$ away from the origin at times $k_{\max}-(i-1),...,k_{\max}+i$ giving, $V \leq \frac{m}{2(n+ \sum k_{i})}$
\end{itemize}
Hence we have $V= \frac{m}{2(n+k)}$.

State of proof: Needs reworking to be more accessible (It also covers the $S_{n}^{k}$ case in Subsection 1.1)

\subsection[]{Proposition for $m=2k_{\max}+1 , 2k_{\max}$}
A proposed similarity is drew to Subsection 1.2. By augmenting the `time-delayed' attack and by using a Combinatorial Improvement (which needs to be developed). The similar logic should hold for the `time-delayed' augmentation, but the Combinatroial Improvement may not yield the same results.

The main issue with getting this is how to change the Combinatorial Improvement to factor in the variety of branches. We may have multiple sets of $M$ type nodes and $R$ type node depending on the length of branches. A set may exist for each branch. Then we need to decide to how to use end-ensuring cycles ?

One suggestion might be to use one on each node where return to the centre is not possible, i.e Has some $M$ or $R$ , then the ones in which we can include in the choosing we `bundle' together to use the minimum number of end-ensuring cycles.

\subsection[]{$S_{n}^{\underbrace{(k,...,k)}_{n \text{ times}}}$}
We will use the fact that the graph is bipartite with some number in the biggest group, depending on if $k$ is odd or even

\begin{itemize}
\item If $k$ odd:
Then we get $n \times \frac{k+1}{2}$ along the branches and $1$ from the centre. Getting $\frac{n(k+1)+2}{2}$ so $V \leq \frac{m}{n(k+1)+2}$.

Note. This does not align well when $m \geq 2(k+1)$ as we are getting a much weaker upper bound here.

\item If even:
Then we get $n \times (\frac{k}{2}+1)$ along the branches (none in centre). Getting $\frac{n(k+2)}{2}$ so $V \leq \frac{m}{n(k+2)}$.

Note. This aligns perfectly when $m \geq 2(k+1)$ with $\sum k_{i}=\sum k=nk$ , providing the attacker a differnt attack
 
\end{itemize}
  
\section[]{Summary on $S_{n} \tightoverset{\scalebox{0.8}{$l$}}{-} S_{n}$}

\subsection[]{$l=0$}
For $l=0$, i.e two immediately joined stars.

The strategies depend on whether the patroller is able to win one game or not and then move to another game and try to win more.

\begin{itemize}
\item If it is possible to do so i.e $m \geq 2n+1$???
  \begin{itemize}
  \item The patroller can use the hamiltonian cycle and achieve $V \geq \frac{m}{2(2n+1)}$
  
  \item The attacker can use a `time-delayed' type attack, attacking external nodes at times $0,1,..,2n$ (making $n(2n+1)$ attacks in total) this gives a bound of $V \leq \frac{m}{2(2n+1)}$
  \end{itemize}

\item If it not possible to do so i.e $m \leq 2n$???
 \begin{itemize}
 \item The patroller can use decomposition to split into $S_{n}$ , $S_{n}$ getting $V \leq \frac{m}{4n}$
 
 \item The attacker can use simplification to $S_{2n}$ getting $V \leq \frac{m}{4n}$
 \end{itemize}
\end{itemize}

State of proof: For the case of $m \geq 2n+1$, proof is needed for the attackers bound
\subsection[]{General $l$}

\begin{itemize}
\item The patroller can use the hamiltonian cycle and achieve $V \leq \frac{m}{2(2n+l+1)}$

\item 
\end{itemize}

%End of main part of document
\bibliography{mybib}

\appendix
\pagenumbering{roman}
\appendixpage
\addappheadtotoc
\section{First appendix section}

\end{document}

